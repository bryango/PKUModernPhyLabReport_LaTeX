%Chinese
	\usepackage[UTF8,fontset=fandol]{ctex}
	\xeCJKsetup{underdot = {
		boxdepth=0pt, format=\huge, depth=.4em
	}}
%	\usepackage[datesep=/]{datetime2} % Use default
	\DeclareTextFontCommand{\textbf}{\sffamily}
%Presenting
	\usepackage[table]{xcolor}
	\usepackage{graphicx}
	\usepackage[space]{grffile}
	\usepackage[font=small,format=plain,%
		labelfont=bf,textfont=it,%
		singlelinecheck=false]{caption}
	\usepackage[above]{placeins}
%	\usepackage{float} % Cause trouble for table footnotes
	\usepackage{wrapfig}
	\usepackage{tabularx,array,booktabs,multirow,bigstrut}
	\newcolumntype{C}[1]{>{\hsize=#1\hsize%
		\centering\arraybackslash}X}
	\newcommand{\minitab}[2][l]{%
		\begin{tabular}{#1}#2\end{tabular}}
	\usepackage{setspace,dcolumn}
	\usepackage{subfig}
	\usepackage{psfrag,epsfig}
%MathSetting
	\let\latexointop\ointop
	\usepackage{amsmath,bm,amssymb,esint,extarrows}
	\usepackage{upgreek,textcomp,mathrsfs}
	\usepackage[only,sslash]{stmaryrd}
	\usepackage{nicefrac,eqnarray}
%	\usepackage{amsthm} % Enable when necessary
%	\usepackage[mathscr]{eucal} % Enable when necessary
	\usepackage{mathtools,physics,siunitx}
	\usepackage{stackengine,varwidth}
	\usepackage{tikz}
	\usepackage{resizegather,empheq}
	\usetagform{default}
	\usepackage{calligra,fourier-orns}
	% Keep \oint unchanged by esint
	\let\ointop\undefined
	\let\ointop\latexointop
	% Define a scriptr 
	\DeclareMathAlphabet{\mathcalligra}{T1}{calligra}{m}{n}
	\DeclareFontShape{T1}{calligra}{m}{n}{<->s*[2.2]callig15}{}
	\newcommand{\scriptr}{\mathcalligra{r}\,}
	\newcommand{\rvector}{\pmb{\mathcalligra{r}}\,}
	% Useful shorthand
	\DeclarePairedDelimiter\ave{\langle}{\rangle}
	\newcommand\inlineeqno{\stepcounter{equation}\ (\theequation)}
	\newcommand{\sinc}{\operatorname{sinc}}
	\newcommand{\mbb}[1]{\mathbb{#1}}
	\newcommand{\mrm}[1]{\mathrm{#1}}
	\newcommand{\mcal}[1]{\mathcal{#1}}
	\newcommand{\tup}[1]{\textup{#1}}
	% Scaling and positioning
	\newcommand\scalemath[2]{\scalebox{#1}{\mbox{\ensuremath{\displaystyle #2}}}}
	\newcommand\raisemath[2]{\raisebox{#1\depth}{${#2}$}}
	\empheqset{box=\nicebox}
	% Presenting
	\newcommand*\nicebox[1]{\fbox{\hspace{1em}\addstackgap[5pt]{#1}\hspace{1em}}}
	\sisetup{%
		redefine-symbols=false,%
		separate-uncertainty=true,%
		range-phrase=\,\textasciitilde\,,%
		arc-separator = \,}
	\allowdisplaybreaks[2]
%ParagraphSetting
	\usepackage{setspace}
	\setlength{\parskip}{.3\baselineskip}
	\usepackage[defaultlines=2,all]{nowidow}
	\postdisplaypenalty=50
%PageSetting
	\usepackage{titlesec}
	\titleformat*{\section}{\large\bfseries}
	\usepackage[colorlinks=true,linkcolor=blue]{hyperref}
	\newcommand{\texstringonly}[1]{%
		\texorpdfstring{#1}{}}
	\usepackage[vmargin={3.5cm,4cm},hmargin=3cm,%
		footnotesep=\baselineskip]{geometry}
%	\usepackage[bottom]{footmisc} % Cause trouble for table footnotes
	\usepackage{changepage}
	% Autoref names
	\renewcommand{\tableautorefname}{\tablename}
	\renewcommand{\figureautorefname}{\figurename}
	% List settings
	\usepackage{enumitem}
	\setlist{itemsep=0pt,topsep=0pt,labelindent=\parindent,leftmargin=0pt,itemindent=*}
	% Some redefined lengths
	\setlength{\headsep}{1.6\baselineskip}
%	\setlength{\footnotesep}{3\parskip} % Use when necessary
	% Header
	\usepackage{fancyhdr,lastpage}
	\pagestyle{fancy}
%	\fancyhf{} % Clear default settings; disabled for now
	\cfoot{--\ \thepage\,/\,\pageref{LastPage} \ --}
	\setlength{\footskip}{2\baselineskip}
	\renewcommand{\headrulewidth}{0.1pt}
	\renewcommand{\headrule}{
		\ifnum\value{page}=1\relax\else
			\vbox to 2pt{
			\hbox to \headwidth{\dotfill}\vss}
		\fi}
	\fancypagestyle{titlepagestyle}{%
		\fancyhead{}
		\chead{
			\vspace{2.5\baselineskip}
			\includegraphics[width=.75\linewidth]{../PKUPhy}}
	}
	% Separator
	\newcommand{\newparagraph}{\pagebreak[3]\noindent%
		\hfil
		~\raisebox{-4pt}[10pt][10pt]{\decofourright~~~~~~~~\decofourleft}~ %
		\par
	}
%	% Background % Use when necessary
%	\usepackage{background} %Waterstamp package
%	\SetBgContents{...的实验报告} %Waterstamp to prevent copying
%	\SetBgScale{5} %Waterstamp setting
	% Essay format
	\renewcommand\appendixname{附录}
	\renewcommand\abstractname{}%摘要
	\renewcommand\tablename{表}
	\renewcommand\figurename{图}
	\renewcommand\refname{参考文献}
	\makeatletter
	\def\@pacs@name{\songti\zihao{-4}{\bf PACS码:}}
	\def\@keys@name{\songti\zihao{-4}{\bf 关键词:}}
	\def\Dated@name{日期:}
	\def\Received@name{\zihao{-5}{接收} }
	\def\Revised@name{\zihao{-5}{修订} }
	\def\Accepted@name{\zihao{-5}{采纳} }
	\def\Published@name{\zihao{-5}{发表} }
	\makeatother
	\linespread{1.5}
	\renewcommand{\labelenumi}{\alph{enumi}.}
	\leftmargini=20mm
	\newcommand{\supercite}[1]{\textsuperscript{\,%
		[\citenum{#1}]}}
	\let\fancycite\cite
	\renewcommand{\cite}[1]{\textup{\fancycite{#1}}}
